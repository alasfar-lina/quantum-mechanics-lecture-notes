\chapter{Summery of Hilbert spaces}
    \section{Representations of Hilbert space}
   We list a summery for the most basic properties of Hilbert spaces, in both representations. \textbf{Discrete} and \textbf{continuous}\\
   \bigskip
   The list goes as follows: \\ 
   \begin{tabular}{c c c}
   	Continuous representation & & Discrete representation \\
   	\hline
   	&&\\
   	$| \psi \rangle = \int \psi (x) |x\rangle dx$ & & $ |\psi \rangle =\sum_n \psi_n |e_n\rangle$\\
   	&&\\
   	$ \langle x'| x\rangle = \int dx \delta (x' -x)$ && 	$ \langle e_m| e_n\rangle= \delta_{mn}$\\
   	&&\\
   	$ \langle \psi | \psi \rangle = \int \psi^*(x) \psi(x) dx =1 $ && 	$ \langle \psi | \psi \rangle = \sum_n | \psi_n|^ 2 = 1$\\
   	&&\\
   		$ \langle \phi | \psi \rangle = \int  \langle \phi | x \rangle \langle x | \psi \rangle dx = \int \phi^* (x) \psi(x) dx$&& 	$ \langle \phi | \psi \rangle = \sum_n \langle \phi | e_n \rangle \langle e_n | \psi \rangle = \sum_n \phi^*_n \psi_n$\\
  &&\\
  
   \end{tabular}
  
   Note that we sometimes denote the basis $ | e_i\rangle$ by $ | i\rangle$.  Here we consider the vector $ | \psi\rangle$ to be normalised.
   
   \section{Operator properties}
   We start by defining the normalised ket :
   \begin{equation*}
   | \psi\rangle = \sum_i \psi_i | i\rangle
   \end{equation*}
   With the property :
   \begin{equation*}
   | \psi_j|^ 2 \equiv P (j) 
   \end{equation*}
   Hence the square modulus of the component is equal to  what-to-be considered as probability. Note that any of the following argument will  also apply to continuous basis representation according to the list above.
   We can write an operator $\hat{ \Omega}$ as :
   \[
   \hat{ \Omega} = \sum_i \omega_i | \omega_i\rangle \langle \omega_i|
   \]
   This is knows as the\textbf{ spectral decompositio}n , where $\omega_i$ and $| \omega_i\rangle$ are the eigenvalues and eigenbasis respectively.
   The expected-value for $ \hat{ \Omega}$ is written as:
   \[
   \langle\hat{ \Omega}\rangle = \langle \psi|\hat{\Omega}| \psi \rangle
   \]
   Proof:\\
   Expanding the above expression :
   \[
 \left(   \sum_k \psi^*_k \langle \omega_k| \right)  \left(   \sum_j \omega_j | \omega_j \rangle \langle \omega_j |\right) \left( \sum_i \psi_i | \omega_i\rangle \right) =
   \]

   \[
 \Leftrightarrow  \sum_{k} \sum_j \psi^*_k \omega_j \underbrace{\langle \omega_k|\langle \omega_j \rangle}_{= \delta_{kj}} \;  \sum_{j}\sum_i \psi_i \underbrace{ \langle \omega_j |\langle \omega_ i \rangle}_{= \delta_{ij}}=
   \]   \[
\Leftrightarrow  \sum_j \omega_j \underbrace{| \psi_j|^ 2}_{= \psi^*_j \psi_j} = \sum_j \omega_j  P(j) \; \; \; \;\Box
   \]
 Similarly, we can define the standard deviation :
 \[
 \sigma^ 2 ( \Omega) = \langle \hat{\Omega}^ 2\rangle - \left( \langle \hat{\Omega}\rangle\right) ^ 2
 \]
 And the expected value for a function of $ \hat{\Omega}$:
 \[
 \langle f( \hat{\Omega})\rangle = \sum_j f(\omega_j) P(j)
 \]
 That implies that the eigenvalue of the function of the operator is the function of the eigenvalue itself.\\
 We now attempt to find the matrix element for an operator $ \hat{\Omega}$ not expressed in the eigenbasis. Starting by :
 \[
 | \phi \rangle =  \hat{\Omega} |\psi \rangle 
 \]
 expanding this expression:
 \[
 \sum_k \phi_k | k \rangle = \hat{\Omega} \sum_i \psi_i | i\rangle 
 \]
 Taking the jth component of the LHS, by projecting on the basis $ | j \rangle$:
 \[
 \phi_j = \sum_i \psi_i \langle \hat j|{\Omega}| i\rangle
 \] 
 Moreover, the ket $| \phi \rangle$ is written as:
 \[
\sum_j \phi_j = \sum_j\sum_i \psi_i \langle \hat j|{\Omega}| i\rangle
 \] 
 We identify the matrix elements of the operator by
 \[
 \Omega_{ji} = \sum_j\sum_i  \langle \hat j|{\Omega}| i\rangle
 \]
 We can simply make the transition to the eigenbasis :
 \[
 \Omega_{ij} = \langle \omega_i| \hat{\Omega}| \omega_j\rangle
 \]
 But $ \hat{\Omega}| \omega_j\rangle = \omega_j | \omega_j \rangle $ , then we have , by linearity:
 \[
  \Omega_{ij} = \omega_j \langle \omega_i|  \omega_j\rangle =
 \]
 \[
 \Leftrightarrow \omega_j \delta_{ij}
 \]
 Hence the operator is diagonalised by the eigenbasis.
 




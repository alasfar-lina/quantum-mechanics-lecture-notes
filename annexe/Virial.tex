\chapter{The Virial Theorem}

Given a system having $ \alpha$ particles, with associated position $\vec{r}_\alpha$ and momenta $\vec{p}_\alpha$. We define the virial function as:
\begin{equation}
\varsigma = \sum_{\alpha} \vec{p}_\alpha \cdot \vec{r}_\alpha
\end{equation}
      It would be interesting to look for the time derivative of this function:
      \begin{equation}
     \dfrac{d \varsigma }{dt} = \sum_{\alpha} \vec{\dot{p}}_\alpha \cdot \vec{r}_\alpha + \vec{p}_\alpha \cdot \vec{\dot{r}}_\alpha
      \end{equation}
    Since we are dealing with many-particle system. We can take the time average for the previous expression 
      \begin{align}
     \langle \dfrac{d \varsigma }{dt} \rangle  =& \frac{\int _0 ^\tau \dfrac{d \varsigma }{dt} dt}{\int _0 ^\tau   dt }\\ \nonumber 
     &= \frac{ \varsigma(\tau)- \varsigma(0)}{\tau}
  \end{align}
  Now, if the system has a periodic motion of a period $ \tau$. The time average for the derivative of the virial function will vanish. even if the system does not admit a periodic motion, the virial function ought to be bounded, hence one can integrate over a sufficiently large interval such that the time average $ \langle \dfrac{d \varsigma }{dt} \rangle $ will approach zero. Hence, we have ( at least as an approximation): \marginpar{ recall that $\vec{F} = \vec{\dot{p}}$}
  \begin{equation}
 \langle \sum_{\alpha}  \vec{p}_\alpha \cdot \vec{\dot{r}}_\alpha\rangle = -  \langle \sum_{\alpha} \vec{\dot{p}}_\alpha \cdot \vec{r}_\alpha \rangle
  \end{equation}
  We can now identify the LHS being twice the kinetic energy , the RHS is the force dotted with the position :
   \begin{equation}
  \langle T \rangle= - \frac{1}{2} \langle \sum_{\alpha} \vec{F}_\alpha \cdot \vec{r}_\alpha \rangle
  \label{vir}
   \end{equation}
   This is the\textbf{ Virial theorem }, the expected value for the kinetic energy for a system is equal to its virial function.\\
   It is interesting to look at forces that arise from central potential taking the form :
 \begin{equation}
 V = k r ^{n+1}
 \end{equation}
 Hence, by the virial theorem eq \eqref{vir}:
 \begin{align}
  \langle T \rangle= & \frac{1}{2} \langle   r \cdot  \frac{d}{d r} (k r ^{n+1} ) \rangle \\ \nonumber
  =& \frac{1}{2}\langle (n+1) k r^{n+1}\rangle \\ \nonumber
  &= \frac{n+1}{2}\langle V \rangle
 \end{align}
 For Columb and gravitational potentials, $ n =-2$. Therefore we have :
 \begin{equation}
 \langle T \rangle= - \frac{1}{2}\langle V \rangle
 \end{equation}
